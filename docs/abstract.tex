\documentclass{article}

\usepackage[utf8]{inputenc}
\usepackage{xspace}
\usepackage{amsmath,amssymb}
\newcommand{\DD}{\mathbb D}
\newcommand{\RR}{\mathbb R}
\DeclareMathOperator{\NN}{\mathbb N}
\DeclareMathOperator{\C}{\mathcal C}
\DeclareMathOperator{\laplace}{\Delta}
\bibliographystyle{amsalpha}

\newcommand{\p}{\ensuremath{\mathcal P}\xspace}
\newcommand{\np}{\ensuremath{\mathcal{NP}}\xspace}
\newcommand{\fp}{\ensuremath{\mathcal{FP}}\xspace}
\newcommand{\sharpp}{\ensuremath{\# \mathcal{P}}\xspace}
\newcommand{\cc}{\texttt{C++}\xspace}
\newcommand{\irram}{\texttt{iRRAM}\xspace}

\begin{document}
\section*{Types for multidimensional functions in \irram}
Numerical Engineering largely employs `recipes' whose correctness and efficiency is demonstrated empirically.
Real Computability and Complexity Theory employ methods from theoretical computer science to give a
While concepts and tools from Theoretical Computer Science are regularly applied,
and significantly support, software development for discrete problems,

We advertise Real Complexity Theory as resource-oriented algorithmic foundation to computations over continuous universes in the bit-model by approximation up to given absolute error with sound semantics,
closure under composition, and proofs of optimal runtime.

Norbert Müller's \cc library \irram provides an implementation of Real Complexity Theory and makes error-free computations with real numbers possible \cite{Mueller00}.
Going one step further, the current work deals with extending the \irram library by types for multidimensional functions $f: \RR^d \to \RR$ and operators on such functions.

However, results from real complexity suggest that many basic operations on functions are computationally hard. 
For example, parametric maximization corresponds to $\p$ vs. $\np$ \cite{MR666209} and integration to the stronger $\fp$ vs. $\sharpp$ \cite{MR748898}.
This remains true if one restricts to smooth functions.
However, the situation improves drastically if only analytic functions are considered: Many operators that are hard in the general case become computable in polynomial time \cite{Kawamura2012}.
It is therefore desirable, to restrict to analytic functions whenever possible.

Analytic functions can be locally represented by a power series around some point.
For uniform computability, however, knowledge of the power series alone is not sufficient.
The data type has to be enriched by some additional information, encoding e.g. the radius of convergence and the maximum of the function in its domain.

The current work gives a \cc implementation of a data-type for multi-dimensional analytic functions based on the \irram library.
It provides efficient implementations of the following operations: Addition, Subtraction, Multiplication, Division, Function Composition, Parametric Maximization, Partial Derivatives and Integration.
Further it is possible to solve initial value problems for ordinary differential equations of the form
$$ \dot y_i(t) = F_i(t, y_1(t), \dots, y_d(t))\, , y_i(t_0) = w_0\, , i = 1, \dots, d$$
where $F_i : \RR^{d+1} \to \RR$ are analytic.


\bibliography{bib}{}
\end{document}

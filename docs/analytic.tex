\documentclass[]{article}

% packages
	\usepackage[utf8]{inputenc}
	\usepackage[thmmarks]{ntheorem}
	\usepackage{xspace,cleveref,xcolor}
	\usepackage{amsmath,amssymb}
	\usepackage{graphicx,url}
	\usepackage[font=small]{caption}
% new commands
	\newcommand{\QQ}{\mathbb Q}
	\newcommand{\RR}{\mathbb R}
	\newcommand{\CC}{\mathbb C}
	\newcommand{\NN}{\mathbb N}
	\newcommand{\ZZ}{\mathbb Z}
	\newcommand{\A}{\mathcal A}
	\newcommand{\B}{\mathcal B}
	\newcommand{\C}{\mathcal C}
	\newcommand{\D}{\mathcal D}
	\newcommand{\R}{\mathcal R}
	\newcommand{\M}{\mathcal M}
	\newcommand{\laplace}{\operatorname\Delta}
	\bibliographystyle{plain}

	\newcommand{\p}{\ensuremath{\mathcal P}\xspace}
	\newcommand{\np}{\ensuremath{\mathcal{NP}}\xspace}
	\newcommand{\fp}{\ensuremath{\mathcal{FP}}\xspace}
	\newcommand{\abs}[1]{\left|#1\right|}
	\newcommand{\sharpp}{\ensuremath{\# \mathcal{P}}\xspace}
	\newcommand{\code}{\texttt}
	\newcommand{\cc}{\code{C++}\xspace}
	\newcommand{\irram}{\code{iRRAM}\xspace}
	\newcommand{\MPFR}{\code{MPFR}\xspace}
	\newcommand{\analytic}{\code{ANALYTIC}\xspace}
	\newcommand{\powerseries}{\code{POWERSERIES}\xspace}
	\newcommand{\real}{\code{REAL}\xspace}
	\newcommand{\temp}{\textcolor{red}}
	\newcommand{\seq}{\mathbf}
	\DeclareMathOperator{\lb}{lb}
	\DeclareMathOperator{\bigo}{O}
	\newcommand{\sprec}{prec\xspace}
	\newcommand{\demph}{\textbf}
	\newcommand{\sdzero}{\texttt{0}}
% ntheorem environments
	\theoremseparator{:}
	\theorembodyfont{\itshape}
	\newtheorem{definition}{Definition}
	\theorembodyfont{\upshape}
	\theoremsymbol{\ensuremath{\blacksquare}}
	\newtheorem{theorem}[definition]{Theorem}
	\newtheorem{corollary}[definition]{Corollary}

\title{Multidimensional Analytic Functions Algorithm Description}
%\author{Akitoshi Kawamura, Florian Steinberg, Holger Thies}

\begin{document}
\section{Overview}
\code{ANALYTIC<d,T>} is a type for analytic functions $f: T^d \to T$.
Currently, $T$ is always supposed to be \code{REAL}.

More specific, \code{ANALYTIC<d,T>} can be used to represent functions analytic on a polydisc $\prod_{i=1}^d \overline B(0, r_i)$ around $0 \in \CC^d$. 
\section{Additional Information}
For most operations to be computable, some additional information has to be supplied.
The information chosen is lower bounds $r_1, \dots, r_d$ on the radius of convergence and an upper bound $M$ for $\{f(x) \, | \, x_i \in \overline{B}(0, r_i) \subseteq \CC \}$.  
Note, that in the current implementation $r_1 = \dots = r_d =: r$.
The additional parameters are of type \code{REAL}.
\section{Constructors}

\section{Evaluation}
To evaluate the function at some point $x$ the sum
$$ \sum_{i_1, \dots, i_d \in \NN^d} a_{i_1, \dots i_d} \cdot x_1^{i_1} \cdot \dots x_i^{i_d} $$
has to be approximated with error less than $2^{-n}$.

Only a finite initial segment
$$ \sum_{0 \leq i_1 \leq N_1, \dots, 0 \leq i_d \leq N_d } a_{i_1, \dots i_d} \cdot x_1^{i_1} \cdot \dots x_i^{i_d} $$
is computed.
The error then consists of two parts
\begin{enumerate}
  \item The error due to using only approximations to the coefficients $a_{i_1, \dots, i_d}$ and the $x_i$.
  \item The truncation error, due to only considering the first $N_i$ terms of each sum.
\end{enumerate}
The evaluation algorithm is recursive in the dimension d.
The procedure \code{evaluate(x, a, B, $q_1$, $\dots$, $q_d$)} takes the point $x$ and the $d$-dimensional sequence $a$, as well as parameters $B$ and $q_i$ s.t.
$$ a_{i_1, \dots, i_d} \leq \frac{B}{q_1^{i_1} \cdot \dots \cdot q_d^{i_d}} $$
Note, that those parameters can be chosen as $B=M$ and $q_i = r_i$ with $M$ and $r_i$ as above as can be shown by using Cauchy's integral formula.
Now let $b_{i_1} := \sum_{i_2, \dots, i_d \in \NN^{d-1}} a_{i_1, \dots, i_d} \cdot x_2^{i_2} \dots x_d^{i_d}$.
Then $f(x_1, \dots, x_d) = \sum_{i \in \NN} b_i\cdot x^i$.
Summing only up to $N$, the total error consists of the truncation error and the error one gets for only having approximations to $b_i$ and $x_i$.
The truncation error can be approximated as follows
\begin{eqnarray*}
\abs{b_i} &\leq& \sum_{i_2, \dots, i_d \in \NN^{d-1}} \abs{a_{i_1, \dots, i_d}} \cdot \abs{x_2^{i_2}} \dots \abs{x_d^{i_d}} \\
&\leq& \sum_{i_2, \dots, i_d \in \NN^{d-1}} \frac{B}{q_1^{i} \cdot \dots \cdot q_d^{i_d}} \cdot \abs{x_2^{i_2}} \dots \abs{x_d^{i_d}} \\
&\leq& \frac{B}{q_1^i \cdot (1-\frac{x_2}{q_1}) \dots (1-\frac{x_d}{q_d})}
\end{eqnarray*}
And
$$ \abs{ \sum_{i=N+1}^\infty b_i x_1^i } \leq \frac{B}{(1-\frac{\abs{x_1}}{q_1}) \dots (1-\frac{\abs{x_d}}{q_d}) }\cdot \left(\frac{\abs{x_1}}{r_1} \right)^{N+1} $$
Now let $a^{i}_{i_1, \dots, i_{d-1}} = a_{i, i_1, \dots, i_{d-1}}$.
Then $\abs{a^{i}_{i_1, \dots, i_{d-1}}} \leq \frac{B}{q_1^{i} \cdot q_2^{i_1} \dots \cdot q_d^{i_{d-1}}}$.
Thus $b_i$ can be approximated by \code{$b_i$ = evaluate($x_2$, $\dots$, $x_d$, $a^{(i)}$,$\frac{B}{q_1^i}$, $q_2$, $\dots$, $q_d$)}
\section{Arithmetic Operators}
\subsection{Addition}
$$a^{(f+g)}_{i_1,\dots,i_d} = a^{(f)}_{i_1,\dots,i_d}+a^{(g)}_{i_1,\dots,i_d}$$
$$r_i^{(f+g)} = \min(r_i^{(f)}, r_i^{(g)}) $$
$$M^{(f+g)} = M^{(f)}+M^{(g)}$$
\subsection{Multiplication}
Define convolution $a * b$ of $d-dimensional$ sequences $a$, $b$ by 
$$ (a * b)_{i_1, \dots, i_d} = \sum_{i=0}^{i_1} (a_i * b_{n-i})_{i_2, \dots, i_d} $$
where $a_i$ is the $d-1$-dimensional sequence obtained when fixing the first parameter and $*$ is the usual multiplication for $d=0$.
Then
$$a^{(f \cdot g)}_{i_1,\dots,i_d} = (a^{(f)}*a^{(g)})_{i_1, \dots, i_d}$$
$$r_i^{(f \cdot g)} = \min(r_i^{(f)}, r_i^{(g)}) $$
$$M^{(f \cdot g)} = M^{(f)} \cdot M^{(g)}$$
\subsection{Division}
If $a_0 \neq 0$ (i.e. $f(0) \neq 0$, A powerseries for $\frac{1}{f}$ can be computed the following way:

Assume $a_{i_1,\dots, i_d}$ is the power series for $f$. 
A powerseries $b_{i_1,\dots, i_d}$ for $\frac{1}{f}$ can be computed by
\begin{eqnarray*}
b_{0, i_2, \dots, i_d} &=& \left(\frac{1}{a_0}\right)_{i_2, \dots, i_d} \\
b_{i+1, i_2, \dots, i_d} &=& \left( -\frac{1}{a_0} \cdot \sum_{j=1}^{i+1}a_j \cdot b_{i+1-j}\right)_{i_2,\dots,i_d}
\end{eqnarray*}
To find suitable parameters $M'$ and $r'$, $r'<r$ has to be chosen s.t. $f(x) \neq 0$ for all $x_i \in B(0, r')$. 
The mean value theorem gives for every $y \in \prod_{i=1}^d B(0, r)$
$$ \abs{f(0)-f(y)} \leq \abs{\nabla f(\xi) \cdot y} \leq d \cdot \frac{Mr'}{r}$$
and thus
\begin{equation*}\label{divlowerbound}
\abs{f(y)} \geq \abs{f(0)}-\abs{f(0)-f(y)} \geq \abs{f(0)}-d\cdot \frac{Mr'}{r} =: L.
\end{equation*}
One can then choose arbitrary $r'$ s.t. $L > 0$ and set $M' = \frac{1}{L}$
\subsection{Power}
At the moment implemented by repeated squaring.
\section{Composition}
$$ F \circ G (x) = \sum_{n=0}^\infty c_n x^n $$
with
$$ c_n = \sum_{k=0}^n a_k \sum_{j_1+ \dots + j_k = n} \prod b_{j_k}$$
Dynamic Programming can be used to compute this.
Let
$$ B_{k, n} = \sum_{j_1 + \dots + j_k = n} \prod b_{j_k}$$
it is
\begin{eqnarray*}
  B_{0,0} & = & 1 \\
  B_{0, n+1} &=& 0 \\
  B_{k+1, n} &=& \sum_{j=0}^n b_j \cdot B_{k, n-j}
\end{eqnarray*}
and
$$c_n = \sum_{k=0}^n a_k \cdot B_{k,n}$$
\section{Differentiation}
Compute the partial derivative $\frac{\partial^k}{x_j^k}f$.
Let $a$ be the power series for $f$ and $b$ the powerseries for the derivative.
Then
$$ b_{i_1,\dots, \i_d} = (i_j+1) \cdots (i_j+k) \cdot a_{i_1, \dots, i_j+k, \dots, i_d} $$
For some $r' < r$, Cauchy's Differentiation formula gives a bound on the derivative:
$$\abs{\frac{\partial^k}{x_j^k}f} \leq \frac{Mk!}{(r-r')^k}$$
Thus for example $r'=\frac{r}{2}, M'=2^k\frac{Mk!}{r^k}$ can be chosen accordingly.
\section{IVP solving}
Consider Systems of Ordinary Differential Equations of the form
\begin{eqnarray*}
  \dot y_v(t) &=& F_v(t, y_1(t), \dots, y_t(t)) \\
  y_v(0) &=& 0 
\end{eqnarray*}
for $v=1,\dots,d$ and all $F_v : \RR^{d+1} \to \RR$ analytic.
The solutions $y_v : \RR \to \RR$ are analytic.
Let $c_v$ be the coefficient sequence for $F_v$.
The coefficient sequence $a_v$ for $y_v$ can then be comptued by the following recursion
\begin{eqnarray*}
  a_{v,0} &=& 0 \\
  a_{v, l+1} &=& \frac{1}{l+1} \sum_{n_1+\dots n_d + k = l} \sum_{\substack {0 \leq i_1 < n_1 \\ \dots\\ 0 \leq i_d < n_d}} c_{v,k,i_1,\dots,i_d} a_{1,n_1}^{(i_1)}\cdot \dots \cdot a_{d, n_d}^{(i_d)}
\end{eqnarray*}
where
\begin{eqnarray*}
 a^{(0)}_{v,0} & = & 1 \\
 a^{(0)}_{v,n+1} & = & 0 \\
 a^{(i+1)}_{v,n} & = & \sum_{j=0}^n a_{v,j} \cdot a_{v, n-j}^{(i)}
\end{eqnarray*}
Let $r_v$, $M_v$ be the parameters for the functions $F_v$ as above and let $M = \max M_v$ and $r = \min r_v$.
By the Picard-Lindel\"{o}f Theorem the above solution is valid for all $t \in [-r', r']$ with $r' = \min(r, \frac{r}{M})$.
Further $r$ is a upper bound for $\abs{y_v}$ on $B(0, r')$. 
\section{Comparing Integration Methods}
We compare several methods to solve ODEs
\subsection{Taylor Series Method}
See above
\subsection{Euler Method}
Repeat $y(t+h) = y(t)+h*F(t,y)$\\
Error:
$$ \frac{1}{2} h^2 y''(\xi) \leq h^2 \frac{M}{r}$$
for $\xi < {r}{2}$
\section{Computing coefficients}
\end{document}
